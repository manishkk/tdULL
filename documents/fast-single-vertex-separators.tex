\documentclass{article}
\usepackage{amsmath, amssymb}
\title{Fast single vertex separators}
\author{Mees}
\begin{document}
\maketitle
Idea: it seems we cannot get out of generating ``most'' separators. Thus we want to generate
separators more quickly. This is one approach that might gain some time.

Let $G$ be a graph. We treat $G$ as the set of its vertices, using notation like $H \subseteq
G$ for subsets of vertices; we treat those as subgraphs, with the understanding that all
subgraphs are induced.

A set $S \subseteq G$ is a \emph{separator} if $G \setminus S$ is disconnected. If $a, b \in
G$, then $S$ (not containing $a, b$) is called an $a, b$-separator if $a, b$ are in different
components in $G \setminus S$. If furthermore there is no subset $S' \subsetneq S$ such that
$S'$ is an $a, b$-separator, then we call $S$ a \emph{minimal} $a, b$-separator. We call a
separator $S$ a minimal separator if there are $a, b$ such that $S$ is a minimal $a,
b$-separator. We call a set $S$ a \emph{fully minimal} separator if there is no $S' \subsetneq
S$ such that $S'$ is a separator. (Note that a separator may be minimal without being fully
minimal.)

If $X \subseteq G$ is a set, then we write
\[
    N(X) = \{v \in G \mid \exists w \in X(vEw)\} \setminus X
\]
for the \emph{neighborhood} of the set $X$. We write $N(v)$ for $N(\{v\})$. If $S$ is a
separator, then $C(S)$ is the set of all connected components of $G \setminus S$, and $C(S)_a$
is the component containing vertex $a$.

Given a graph $G$ we can make a directed graph whose nodes are all the minimal separators of
$G$, where the outgoing edges from a separator $S$ go towards all the separators of $G$ that
can be created by the following procedure:
\begin{itemize}
    \item Choose a vertex $x \in S$. Create the set $S_x = S \cup N(x)$.
    \item Choose a connected component $H$ in $C(S_x)$.
    \item Then $N(H) \subseteq S_x$ is a minimal separator.
\end{itemize}
Thus, from a minimal separator we can generate new minimal separators by generating all the
$S_x$, and generating a new separator $N(H)$ from each component of $G \setminus S_x$.

Our current algorithm works on this graph: we pick some minimal separators as starting points,
and do a graph traversal (BFS, DFS) to find all minimal separators. We have a theorem saying
that this gives all of them.

The minimal separators we start with are:
\[
    \{N(H) \mid H \in C(N(v) \cup \{v\}), v \in G\}.
\]

(What we actually want are \emph{fully minimal} separators. It is not really important for this
document, but a separator $S$ is fully minimal if and only if for each $H$ in $C(S)$, $N(H) =
S$. You can check this in linear time by finding $N(H)$ for each $H$ and checking that it is
always equal to $S$.)

I propose the following adaptation of this algorithm. Instead of starting from every vertex, we
start only from the vertex $a$. And instead of growing in any direction -- adding all of $N(x)$
for a vertex $x \in S$ -- we grow only ``away from $a$''. I have not yet proved that this will
give us all the separators we need, but my inuition says that it will.

Thus we fix a specific vertex $a$, and we form a graph now consisting only of all minimal $a,
b$-separators, where $b$ is any vertex. Note that this is precisely the set of minimal
separators that do not contain $a$. As our starting separators, we choose only
\[
    \{N(H) \mid H \in C(N(a) \cup \{a\})\}.
\]
Then, the outgoing edges from a separator $S$ are given by the separators you can generate with
the following procedure:
\begin{itemize}
    \item Choose a vertex $x \in S$. Create the set
        \[
            S_x = S \cup \{v \in N(x) \mid v \notin C(S)_a\}.
        \]
    \item Choose a connected component $H$ in $C(S_x)$ that is not $C(S_x)_a$.
    \item Then $N(H) \subseteq S_x$ is a minimal separator.
\end{itemize}
The idea is: because we do not generate ``back-and-forth'', you avoid generating the same
separator as often. And if we keep track of $C(S)_a$ together with the separator $S$, we can
skip some of the DFS-es.

Now, this idea may give some meaningful improvements, but the real interesting part would be
this: if we can use a dynamic connectivity data structure, we can keep track not only of
$C(S)_a$, but in fact of all connected components in $C(S)$, and this saves us almost all of
the DFS-es that we have to do. Then we if we do the graph traversal of our separator graph with
a depth first search, then we can do our search moving up and down the search tree every time,
making only minor changes to the connectivity data structure, which may be very fast. (But
that's going to be, to put it mildly, very complicated, and depends strongly on how effective
the dynamic connectivity data structure is \emph{in practice}.)
\end{document}
